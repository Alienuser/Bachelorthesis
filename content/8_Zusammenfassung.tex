\chapter{Zusammenfassung}
\label{cha:zusammenfassung}

Im Rahmen dieser Arbeit wurde eine Hybrid-Cloud-Architektur entwickelt und diese anschließend durch eine prototypische
Implementierung realisiert. Ziel war es, einem Unternehmen dabei zu helfen, die Vorzüge einer Cloud zu nutzen, ohne auf
die eigene Infrastruktur verzichten zu müssen.

Gerade die Idee hinter einer serviceorientierten Architektur, Programme auf verschiedenen Systemen zu einer Gesamtanwendung
zu verbinden, kann durch die Hybrid-Cloud realisiert werden. Dabei läuft ein Teil der Anwendung auf einem
internen Mainframe und ein anderer Teil kann öffentlich in der Cloud aufrufbar sein.

Basis für die Architektur bildet ein z Systems und Bluemix, die Cloudumsetzung der IBM. Die Wahl fiel zugunsten
dieses Toolsets aus, da die Komponenten sehr weit verbreitet sind und viele offene Standards besitzen.

Am Anfang der Arbeit wurde die Hybrid-Cloud-Architektur entwickelt und im weiteren Verlauf umgesetzt. Die verwendeten
Zusatzprogramme wie z.B. Runtimes wurden mit anderen verglichen und anschließend installiert und eingerichtet.

Nachdem die Architektur aufgebaut war, wurde sie mittels einer prototypischen Implementierung umgesetzt. Teil der Umsetzung
war ein Web-Frontend auf Basis von AngularJS und Material Design sowie zwei Smartphone-Applikationen.

Aus den Erfahrungen, die in der Entwicklung der Architektur und der prototypischen Anwendung gesammelt wurden, ergaben
sich Diskussionsthemen und einige Erkenntnisse, die der Weiterentwicklung des Systems dienen können. Diese wurden in
Kapitel \ref{cha:ausblick} ausgeführt.

Letztendlich bleibt mir zu sagen, dass sich die Hybrid-Cloud, im Gegensatz zur reinen Cloud, in Europa und insbesondere
in Deutschland durchsetzen wird. Der einfache Grundaufbau, die vielseitige Konfiguration und die zahlreichen
Einsatzmöglichkeiten tun ihr Übriges dazu.

Das Fazit der IBM Studie \textit{Tailoring hybrid cloud: Designing the right mix for innovation, efficiency and growth}
besagt abschließend, \glqq [\ldots] dass 80 Prozent der Unternehmen weltweit die Cloud bereits nutzen und zwar meist in
hybrider Form [\ldots]\grqq~\cite{online_ausblick_fazit}.