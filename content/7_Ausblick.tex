\chapter{Ausblick}
\label{cha:ausblick}
In diesem Kapitel sollen weitere Services und Funktionen erwähnt werden, durch die die prototypische Anwendung erweitert
und verbessert werden kann.

Außerdem sollen Möglichkeiten aufgezeigt werden, welche die Arbeit mit der Hybrid-Cloud-Architektur unter Bluemix
und z Systems vereinfacht.

\section{Nutzen von API-Connect}
In der jetzigen Version wird um die COBOL-Anwendung ein Java-Wrapper geschrieben, welcher ein REST-Interface für Anfragen
von außen zur Verfügung stellt. Wenn die COBOL-Anwendung erweitert werden würde oder noch andere Funktionen dieser genutzt
werden sollen, muss immer die Java-Applikation angepasst werden.

Alternativ ist es möglich, den API-Connect Service von Bluemix zu nutzen, welcher eine browserbasierte GUI zur Verfügung
stellt, mit welcher es möglich ist, REST(Ful)-Interfaces zu entwickeln.

Bei der Nutzung des Services würde das Web-Frontend (und die Smartphone-Apps) ihre Anfragen an den Service schicken. Dieser
würde dann über den Secure Gateway eine Verbindung auf die COBOL-Anwendung aufbauen und Informationen abrufen.

Außerdem ist es möglich, dass der API-Connect Service auch den DB2 REST Service aufrufen kann. So hat das Web-Frontend
lediglich eine Schnittstelle, die es ansprechen muss. Der API-Connect Service schickt die Anfragen dann an verschiedene
Anwendungen weiter.

\section{Open Service Broker Projekt}
Da die Schnittstellen eines Servcie Brokers von der \textit{Open Service Broker Projekt} Organisation definiert werden und
dieser zahlreiche große Firmen angehören, welche jeweils eigene Cloud-Lösungen anbieten, kann der entwickelte Service
Broker auch auf anderen Cloud-Infrastrukturen eingerichtet werden.

So könnte der Service Broker zum Beispiel auch bei Amazons AWS instanziiert werden und somit theoretisch eine Verbindung
zwischen dieser Cloud und einer anderen Cloud oder einem Mainframe aufgebaut werden. Allerdings unterstützt AWS bislang
nicht die Möglichkeit VPN Verbindungen zu anderen Netzwerken aufzubauen.

\section{Einbindung in den Katalog}
Aktuell ist der entwickelte Service Broker nur in der Bluemix Organisation über die Cloud Foundry CLI sichtbar in der er
eingerichtet wurde. Dort auch nur für Nutzer, welche auf die Organisation Zugriff haben. Er erscheint jedoch nicht im
Bluemix Katalog.

Das Hinzufügen eines Service Brokers in den allgemeinen Bluemix Katalog kann lediglich durch einen Administrator erfolgen.
Dieser kann allerdings einstellen, in welcher Region oder welchen Kundengruppen dieser Service Broker zur Verfügung stehen
soll.

Beim Entwickeln des Service Brokers wurde dieser schon mit einem Icon und einem Beschreibungstext ergänzt, sodass das
Hinzufügen in den Katalog keine Nacharbeit am Quellcode des Service Brokers bedeuten würde.

Nach weiteren Tests könnte eine Anfrage an einen Administrator von Bluemix Public gestellt werden, sodass der Service
Broker in den Katalog aufgenommen werden kann. Dies geschieht über das
Support-Formular\footnote{https://developer.ibm.com/answers/questions/ask/?topics=bluemix}.

\section{Weitere zOSMF Interface Funktionen}
Bisher wird im Service Broker neben dem Abruf der allgemeinen Informationen lediglich noch die Funktion zum Ausführen
von Workflows und Jobs der einzelnen Instanzen des zOSMF Interfaces genutzt.

Das Interface von zOSMF bietet allerdings noch weitaus mehr Funktionen.

Eine Möglichkeit wäre das Auslesen der aktuellen Workflows. Aktuell kann lediglich ein Workflow gestartet werden,
welcher eine CICS-Region erstellt. Wenn alle Workflows dynamisch abgefragt werden, könnten neu erstellte auch ausgeführt
werden.

Desweiteren könnten Workflows in Bluemix mit einem visuellen Editor geschrieben werden (ähnlich
Node-Red\footnote{https://nodered.org}) und dann an zOSMF übertragen werden. So könnten, auch automatisiert, neue Workflows
entstehen, welche Runtimes oder Anwendungen auf dem Mainframe erstellen.

Auch könnten die Notifications für einen Benutzer ausgelesen werden. Dadurch wäre es möglich festzustellen, wann der
gestartete Workflow abgeschlossen wurde. Denn immer wenn ein Workflow gestartet, gestoppt, durchgeführt oder in einen
Fehler läuft, wird der Benutzer mit einer Notifications darauf aufmerksam gemacht.

\section{MobileFirst Platform}
Bei den Smartphone-Apps handelt es sich jeweils um native Applikationen, welche ein WebView-Layout beinhalten, in das beim
Start immer die aktuellste Version des Web-Frontends geladen wird.

Die Interaktion mit dem Frontend übernimmt das WebView-Layout, und getätigte Einstellungen werden auf dem Smartphone
persistiert.

Um die MobileFirst Platform zu nutzen, muss der Service in das Web-Frontend eingebunden werden. Die MobileFirst Platform
nutzt dabei Cordova um die Webseite dann in eine native Android, iOS und Windows App zu kompilieren. Dabei werden
HTML Elemente durch die Systemeigenen ersetzt. Ein HTML-Button wird dann zum Beispiel zu einem nativen Button.

Der große Vorteil der Nutzung dieser Platform ist neben der Möglichkeit neben Android und iOS auch Windows Phone zu
unterstützen, dass weiterhin nur ein Web-Frontend geschrieben werden muss und dies dann in eine native Applikation
umgewandelt werden kann. Daraus ergibt sich ein besseres \textit{Look and Feel} und eine höhere Performance
der Applikation.

\section{Native Smartphone-Apps}
Anstatt wie bisher das Web-Frontend in einen WebView-Layout zu laden oder mit Hilfe der MobileFirst Platform native Applikationen
aus HTML zu erstellen, können für die verschiedenen Smartphonebetriebssysteme jeweils eigene, native App geschrieben
werden.

Dies hat, neben vielen weiteren, den Vorteil, dass direkt auf die Hardware des Systems zugegriffen werden kann und Daten
mit anderen Apps ausgetauscht werden können.

So ist es zum Beispiel möglich, Informationen, die in der Anwendung angezeigt werden, mit den Kontaktdaten auf dem Smartphone
abzugleichen und den Datenstand gegenseitig anzupassen.

\section{Erweiterung der Toolchain}
\label{sec:erweiterung_der_toolchain}
Die in diesem Projekt genutzte Toolchain besitzt lediglich eine rudimentäre Verbindung zum Mainframe um dort eine Anwendung
einzurichten. Diese wurde händisch über das FTP-Protokoll aufgebaut. Allerdings gibt es auch für die Toolchain möglichkeiten
eigene Services zu schreiben. Ähnlich dem Service zur Integration eines GitHub Projektes.

Die Funktion der Toolchain könnte dahingehend erweitert werden, dass aus dem instanziierten Secure Gateway Service alle
Verbindungen zu den CICS-Regionen herausgesucht werden und automatisch ein Build-Prozess für diese eingerichtet werden kann.

Somit wäre ein deployment auf die verschiedenen CICS-Regionen einfacher über die Toolchain möglich ohne für jede CICS-Region
selbstständig eine FTP- oder SSH-Verbindung aufzubauen.

\section{Automatischer Secure Gateway}
Wenn über den Service Broker eine neue CICS-Region erstellt wird und diese auch über die Cloud genutzt werden soll, muss
nach der erfolgreichen Einrichtung händisch ein Ziel im Secure Gateway Service hinzugefügt werden.

Einfacher wäre es, wenn es bei der Erstellung einer CICS-Region eine Auswahl mit \textit{Make public} gäbe, welche nach
Auswahl ein Ziel automatisiert hinzufügt.

Um diese Funktionalität zu erreichen, muss der Secure Gateway Service in Bluemix überarbeitet werden. Er muss über ein
REST-Interface verfügen, über das es möglich ist, neue Ziele hinzuzufügen. Als Parameter müsste das Interface zumindest
\textit{Name}, \textit{IP-Adresse} und \textit{Port} unterstützen.

\section{Auslesen der zOSMF-URL}
Um die Hybrid-Cloud-Architektur zu nutzen, muss ein Ziel im Secure Gateway zum zOSMF erstellt werden. Anschließend kann
der Service Broker instanziiert werden. Dieser muss im ersten Schritt mit URL, Benutzername und Passwort für zOSMF
eingerichtet werden.

Einfacher wäre es, wenn die angelegte zOSMF-URL im Secure Gateway Service automatisiert vom Service Broker ausgelesen
werden könnte. Um dies zu realisieren, muss der Secure Gateway Service über ein REST-Interface verfügen, über den alle
eingerichteten URLs ausgelesen werden können.

Anschließend könnte der Service Broker entweder alle URLs anzeigen und der Benutzer wählt die Richtige aus, oder anhand
der Benamung der URL wird dieser vorausgewählt. Denkbar wäre die Benamung \path{zOSMF}.

\section{Automatisches Auslesen von URLs}
Damit das Frontend mit dem Backend auf dem Mainframe kommunizieren kann, muss in Bluemix im Secure Gateway Service ein
Ziel dafür eingerichtet werden. Diese URL muss im Frontend gespeichert werden, damit die Angular-Factories Anfragen an
das Backend schicken können.

Für die Umsetzung wäre es einfacher, wenn der Secure Gateway Service über die VCAP-Variablen alle seine Ziele
zur Verfügung stellen würde. So könnte das Frontend die URL dynamisch aus dem Secure Gateway Service auslesen und die
Anfragen an das Backend schicken.

Ein weiterer Vorteil wäre, dass wenn sich zum Beispiel die CICS-Region ändert eine neue URL hinzugefügt werden kann und
das Web-Frontend automatisch über den Namen das richtige Ziel auswählt ohne ein neues Deployment mit Quellcodanpassungen
durchführen zu müssen.

Auch wäre es so einfacher eine \textit{DEV}- und \textit{PROD}-Stage des Web-Frontend zu nutzen. Entsprechend wären zwei
Ziele im Secure Gateway Service eingerichtet (welche zum Beispiel auf unterschiedliche CICS-Regionen zeigen), und das
Web-Frontend wählt, je nach Stage, die richtige URL aus.

\section{Unterschiedliche Mainframes}
Aktuell wird für jede Konfiguration eines Mainframes ein eigenständiger Service Broker benötigt. Meist unterhält ein
Unternehmen aber mehrere z Systems in seinem Rechenzentrum.

Sinnvoll wäre, dass der Service Broker mehrere Verbindungen zu unterschiedlichen z Systems aufbauen kann (vorausgesetzt
den Verbindungen durch den Secure Gateway Service). Dann könnte bei der Erstellung einer CICS-Region nach dem z System
gefragt werden, auf dem die CICS-Region eingerichtet werden soll.

Auch könnten die \textit{Tenants} für die Auswahl der verschiedenen z Systems genutzt werden.

\section{Fragenkatalog}
Wie in Kapitel \ref{sec:rechte_der_entwickler} auf Seite \pageref{sec:rechte_der_entwickler} beschrieben, müssen dem
Entwickler Hilfen angeboten werden, um sicherzustellen, dass die datenkritischen Anwendungen richtig geschützt werden
können.

Dafür wäre bei der Erstellung einer neuen CICS-Region über den Service Broker ein vorangeschalteter Fragenkatalog hilfreich.
In diesem wird der Benutzer gefragt, um welche Art von Anwendung es sich handelt und um welche Daten es geht. Nach der
Beantwortung der Fragen, kann der Service Broker die Rechenpower in Form einer CICS-Region auf dem Mainframe oder
als zum Beispiel Cloud Foundry Applikation in Bluemix zur Verfügung stellen.

Die Entscheidung, wo die Ressourcen bereit stehen, trifft also nicht der Entwickler sondern der Service Broker.

\section{Hinzufügen einer Applikation}
Bei der Einrichtung einer CICS-Region wird der hinterlegte Workflow genutzt, um den Job durchzuführen. Dabei ist fest
vorgegeben, wie und mit welchem Inhalt die CICS-Region erstellt wird. Ein Abändern des Workflows ist oft mühsam und
setzt technisches Verständnis voraus.

Da CICS-Regionen ohne vorinstallierte Anwendung wenig Sinn machen, könnte bei der Erstellung einer solchen auch gleich
nach einer Anwendung gefragt werden, welche vorinstalliert werden soll. Die Auswahl der Anwendungen könnte aus dem RTC bzw.
UCD kommen (eine Verbindung dazu gibt es schon). Dort könnten alle Verfügbaren Anwendungen ausgelesen und dem  Benutzer
via Auswahl angezeigt werden.

Im nächsten Schritt würde die CICS-Region erstellt werden und im Anschluss über UCD ein Build auf dieses System mit der
ausgewählten Applikation.

\section{Andere Cloudanbieter}
Zur Zeit kann der Service Broker lediglich CICS-Regionen über zOSMF auf einem Mainframe bereitstellen. Oftmals gibt es
in einem Unternehmen neben einem Mainframe auch mehrere Cloudanbieter. Denkbar wäre die Integration eines anderen
Cloudanbieters.

So hätte der Nutzer des Bluemix Service Brokers die Möglichkeit neben dem Erstellen einer CICS-Region auf dem Mainframe
auch das Hinzufügen von Runtimes in anderen Cloud-Umgebungen um eine Anwendung über mehrere Cloud-Anbieter und seinem
Mainframe hinweg zu entwickeln.

Dafür müsste lediglich eine Verbindung zum weiteren Cloud-Anbieter hergestellt werden und dem Service Broker der dortige
Benutzername mit Passwort bekannt gemacht werden.