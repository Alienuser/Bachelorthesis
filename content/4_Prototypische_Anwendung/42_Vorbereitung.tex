\section{Vorbereitung}
In diesem Abschnitt werden Vorbereitungen für die Entwicklung auf Bluemix und für Cloud Foundry Container getroffen. Dabei
werden unter anderem Programme installiert und eingerichtet.

\subsection{Git}
Um mit Git-Repositorys arbeiten zu können, wird ein lokaler Git-Client benötigt. Dieser kann unter Ubuntu mit folgenden Befehlen
installiert und eingerichtet werden.

\begin{lstlisting}[language=bash, caption=Installieren von Git, label=Installieren von Git]
   $ sudo apt-get update
   $ sudo apt-get install git
\end{lstlisting}

Eine Windows- und MacOS-Installer kann auf der Webseite von Git heruntergeladen werden\footnote{https://git-scm.com}.

\subsection{IBM Application Discovery}
Um die COBOL-Anwendung analysieren zu können wird das Tool IBM Application Discovery genutzt. Die Installation des Tools
kann auf zwei Arten geschehen.

\begin{itemize}
    \item{Manuelle Installation des Servers und der Clients}
    \item{Nutzen einer Virtual Machine}
\end{itemize}

Da die manuelle Installation des Servers und der Clients recht langwierig ist und in der
Installationsanleitung\footnote{https://www.ibm.com/support/knowledgecenter/en/SSRR9Q} nachgelesen werden kann, wird hier
kurz auf die Nutzung der Virtual Machine eingegangen.

Um die Virtual Maschine nutzen zu können, wird ein Tool benötigt um diese abzuspielen. Eine Möglichkeit ist das Programm
\path{VM VirtualBox}\footnote{https://www.virtualbox.org} von Oracle.

Nach der Installation von VM VirtualBox muss die Virtual Mashine von IBM Application Discovery heruntergeladen werden.
Die Dateien stehen im IBM
Repository\footnote{http://ausgsa.ibm.com/projects/r/rational\_mktengr\_team/EM/EZSourceSoftwareAndDocs\\/VMImage2016-10-05}
zur Verfügung.

Nach dem Download der 19 Teilarchiv-Dateien müssen diese durch einen Archivmanager zusammengefügt und die \path{.vmx}-Datei
extrahiert werden.

Im letzten Schritt wird die \path{.vmx}-Datei als Festplatte in VM VirtualBox hinzugefügt und gestartet.

In der gestarteten Virtual Mashine befindet sich sowohl der IBM Application Discovery Server als auch ein Client, mit dem
die Anwendungen analysiert werden können.