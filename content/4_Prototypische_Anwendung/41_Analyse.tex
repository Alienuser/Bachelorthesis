\section{Analyse}

\subsection{Architektur}
Für Webseiten gibt es zahlreiche Frameworks, welche wiederum zahlreiche Architekturen umsetzen. Das MVC-Pattern hat sich
in der Vergangenheit großflächig durchgesetzt (siehe dazu \cite{book_prototypischeanwendung_mvc}) und wird in diesem Beispiel
verwendet.

Ein großer Vertreter des MVC-Patterns ist AngularJS. Neben der schnellen Einarbeitung ist AngularJS sehr mächtig und es
kann eine Single-Page Applications (kurz SPA) geschrieben werden. Bei diesem Web-Frontend soll es sich um eine SPA handeln.

Auf der Backendseite wird ein REST-Interface entwickelt, welches die eingehenden Anfragen an die COBOL-Anwendung weiterleitet.
Da seit kurzem ein WebSphere-Liberty-Profile für CICS existiert und dieses neben vielen Vorteilen gegenüber anderen
Application-Servern sich noch großer Beliebtheit erfreut (siehe \cite{online_prototypischeanwendung_cicsliberty}),
wird das REST-Interface in Java entwickelt, um das WebSphere-Liberty-Profile zu nutzen.

Um einen weiteren Vorteil der Hybrid-Cloud-Architektur aufzuzeigen, soll außerdem die REST-Schnittstelle der vorliegenden
DB2-Instanz genutzt werden. Dazu wird ein so genannter Service eingerichtet, der die Daten zurückgeben kann, welche die
COBOL-\-Anwendung standardmäßig nicht zurückgeben kann.

\subsection{Services}
Um einen Mehrwert der Cloud aufzuzeigen, soll das Web-Frontend durch einen Cloud-Service aufgewertet werden. Dabei stehen
neben analytischen Services in IBM Bluemix auch Watson Services zur Verfügung.

Da die Anzahl der Daten in der Demo zu gering ist, entsteht kein großer Mehrwert durch analytische Services. Es wird
deshalb der Watson Service Text2Speech eingebaut. Dieser wandelt geschriebenen Text in Sprache um. Bei der Umwandlung
entsteht eine Audio-Datei, welche in gängigen Webbrowsern abgespielt werden kann.

Der Service wird immer dann verwendet, wenn das Backend Daten an das Web-Frontent zurückgegeben hat.

\subsection{Runtime}
Um das Web-Frontend auf Bluemix laufen zu lassen, wird eine Runtime benötigt. Zur Auswahl stehen neben Cloud Foundry auch
ein Docker-Container und der Service Object Storage.

Der Object Storage Service speichert Dateien online. Diese können dann mittels statischer URL aufgerufen werden. Um eine
AngularJS-Webseite damit zu hosten, müssen alle Dateien hochgeladen und anschließend referenziert werden. Das
ist relativ mühsam, da eine Referenzierung in der lokalen IDE erst nach dem ersten Upload möglich ist.

Alternativ kann in einem Docker Container ein Webserver laufen, welcher das Web-Frontend statisch zurückliefert.

In diesem Beispiel wird ein Cloud Foundry Container verwendet. Dieser hat den Vorteil, dass er Zugriff auf die
Cloud Foundry Environment-Variablen\footnote{https://docs.run.pivotal.io/devguide/deploy-apps/environment-variable.html}
(kurz cf env) in Bluemix hat, welche die Zugriffsinformationen für die verschiedensten Services hält.

Im Cloud Foundry Container läuft ein Express Server unter NodeJS. Dies erleichtert unter anderem den Zugriff auf die
VCAP-Variablen, welche sich in den cf env Variablen befinden.

Das Backend wird in einem WebSphere-Liberty installiert, welches in einem CICS läuft. Es wird dabei das
WebSphere-Liberty-Profile genutzt.

\subsection{Smartphone App}
Eine Möglichkeit die Smartphone-Apps zu erstellen ist es, sie nativ in der jeweiligen Sprache zu schreiben und auf dem
Smartphone zu installieren. Das Problem ist, dass das jeweilige Smartphone dann eine VPN-Verbindung zum Mainframe-Netz
besitzen muss, um auf das Backend zugreifen zu können. Das Öffnen des Mainframe-Netzwerkes für mehrere Endgeräte kann
allerdings zu einer große Sicherheitslücke führen.

Alternativ kann auch der Secure Gateway genutzt werden, welcher schon eine Verbindung in das Rechenzentrum hat. Dann
müsste bei einer Änderung des Secure Gateways allerdings immer ein Update für die entsprechenden Applikationen zur
Verfügung gestellt werden.

Auch könnte eine WebViewer-App entwickelt werden. Dabei wird das Web-Frontend in einen Web-Container geladen und dem
Nutzer suggeriert, dass es sich um eine reale App handelt. Sofern die Webseite so gestaltet ist wie eine App, fällt der
Unterschied in den meisten Fällen nicht auf.

In diesem Beispiel wird eine WebViewer-App entwickelt, da so dass schon geschriebene Web-Frontend wiederverwendet werden
kann und sich dadurch die Entwicklungszeit und Wartung deutlich verkürzt.

\subsection{Frontend-Design}
Genau wie bei der Architektur einer Anwendung gibt es auch im Bereich Design einer Webseite zahlreiche Frameworks. Nach
t3n \cite{online_prototypischeanwendung_cssframework} ist \path{Bootstrap} das wohl bekannteste CSS Framework.

Da in diesem Beispiel allerdings auch Smartphone-Apps erstellt werden, muss das Web-Frontend eine große Ähnlichkeit zu
diesen Apps haben. Aus diesem Grund wird Googles Material Design verwendet. Für AngularJS gibt es das Angular
Material Plugin\footnote{https://material.angularjs.org/latest}, welches die Material-Design
Spezifikationen\footnote{https://material.io/guidelines} von Google umsetzt.

\subsection{Bestehende Anwendung}
Um zu demonstrieren, wie ein neu geschriebenes Web-Frontend in der Cloud mit einer Anwendung auf einem Mainframe
kommunizieren kann, wird eine schon bestehende Anwendung benötigt, welche den Bestand einer Firma repräsentieren soll.

Hierfür wird eine einfach gehaltene COBOL-Anwendung verwendet, welche Daten in eine Datenbank schreiben, diese auslesen
und verändern kann. Im folgenden \path{GenApp} genannt.

Die Schnittstellen und genauen Funktionen der GenApp sind allerdings nicht bekannt und müssen in weiteren Schritten
analysiert werden.

\subsection{Qualitätssicherung}
Damit beim regelmäßigen Bau des Quellcodes durch die Bluemix Toolchain bzw. durch UCD immer die richtige Qualität gewahrt
werden kann, sollen für das Web-Frontend, das Java-Backend und die Smartphone-Apps automatisierte Tests geschrieben werden.

Dabei wird das Web-Frontend durch PhantomJS, einem Headless Browser, getestet. Das Java-Backend und die Android-App jeweils
mit JUnit-Tests.