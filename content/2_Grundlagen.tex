\chapter{Grundlagen}
\label{cha:grundlagen}

Im folgenden Kapitel sollen die elementaren Grundlagen beschrieben werden, die zum Verständnis der nachfolgenden Kapitel
notwendig sind.

Der erste Abschnitt befasst sich mit der reinen Cloud und einer Abwandlung davon, der Hybrid-Cloud.

Der zweite Teil widmet sich den Produkten, die IBM in der Cloud und im Mainframebereich bietet. Dabei wird unter
anderem auf die Konfiguration der Produkte und Laufzeitumgebungen eingegangen wie auch Programme und Konzepte, welche in
der Arbeit benötigt werden, behandelt.

Im letzten Teil des Kapitels folgt die Beschreibung einer Designidee bei Smartphone-Apps und allgemeine Konzepte der
Softwareentwicklung.

\section{Cloud}
Der Begriff Cloud hat sich als Kurzform des Cloud Computing etabliert und versteht das Zusammenspiel von mehreren
Servern. Die Server übernehmen Aufgaben, wie etwa die Datenspeicherung oder komplizierte Programmabläufe. Dabei erkennt
der Cloud-Nutzer nicht, wie viele Server hinter der Cloud stecken oder wo diese sich befinden.

Selbst wenn ein Server ausfällt, hat dies keine Auswirkungen auf das gesamte System, da die Anfragen und Aufgaben auf
die anderen Systeme umgeleitet werden.

Die Cloud zeichnet sich nach NIST \cite{online_grundlagen_cloudNIST} und \cite{online_grundlagen_cloud} durch fünf
wesentliche Eigenschaften aus:

\begin{itemize}
    \item \textbf{On-Demand Self Service}   \\ Registrierte Nutzer können Resourcen selbstständig instantiieren und konfigurieren.
    \item \textbf{Broad Network Access}     \\ Der Zugriff kann von verschiedenen Endgeräten erfolgen.
    \item \textbf{Resource Pooling}         \\ Alle Resourcen des Anbieters werden gebündelt und nach Bedarf den Nutzern zugewiesen.
    \item \textbf{Rapid Elasticity}         \\ Kapazitäten können nach Bedarf skaliert werden und stehen schnell und dynamisch zur Verfügung.
    \item \textbf{Measured Service}         \\ Es existiert eine automatische Kontrolle der Ressourcen durch einen Zähler, welcher die Transparenz für den Anbieter und den Benutzer ermöglicht.
\end{itemize}

Neben vielen Cloud-Anbietern wie Amazon, SAP und Microsoft gibt es IBM mit \textbf{Bluemix}, welches im folgenden als Lösung für
die prototypische Implementierung dient.

\section{Hybrid-Cloud}
Mit Hybrid-Clouds werden Mischformen der public und der private Cloud bezeichnet. So laufen bestimmte Services bei
öffentlichen Anbietern über das Internet, während datenschutzkritische Anwendungen und Daten im Unternehmen betrieben
und verarbeitet werden. Die Herausforderung liegt hier in der Trennung der Geschäftsprozesse in datenschutzkritische
und -unkritische Workflows.

Voraussetzung ist eine saubere und konsequente Klassifizierung der im Unternehmen vorhandenen und verarbeiteten
Daten. Siehe dazu \cite{online_grundlagen_hybriddcloud}.

\section{Bluemix}
Bluemix ist die von IBM entwickelte Cloud-Platform. Über Bluemix greifen Entwickler auf mehr als 160 Cloud-Services zu,
um mobile Apps und Webanwendungen zu entwickeln. In Bluemix gibt es zahlreiche Analysewerkzeuge sowie Services von
Drittanbietern. Mit Watson Analytics lassen sich beispielsweise intelligente Systeme realisieren, die Daten
kognitiv (also selbstlernend, ohne für die Problemlösungen programmiert zu sein) auswerten und für die Entscheidungsfindung
aufbereiten.

Bluemix unterstützt diverse integrierte DevOps-Dienste, um Cloud-Anwendungen zu erstellen, auszuführen, bereitzustellen
und zu verwalten. Die Entwicklerplattform basiert auf der Technologie von Cloud Foundry und läuft auf IBMs
Softlayer-Cloud-Infrastruktur. Sie unterstützt mehrere Programmiersprachen, einschließlich Java, Node.js, Go, PHP, Python,
Ruby Sinatra, Ruby on Rails und kann auch andere Sprachen wie Scala durch den Einsatz von Buildpacks unterstützen
\cite{online_grundlagen_bluemix}.

Weitere Informationen über Bluemix können auf der Webseite\footnote{https://bluemix.net} gefunden werden.

\subsection{Public, Dedicated und Local}
Wie in \cite{online_grundlagen_bluemix_pdl} zu lesen, kann Bluemix in drei verschiedenen Varianten betrieben werden.

\begin{itemize}
    \item Bei \textbf{Public} werden Server, welche öffentlich im Internet zur Verfügung stehen, genutzt. Diese Variante wird gemeinhin als \path{Cloud} bezeichnet.
    \item Bei \textbf{Privat} können hingegen nur Server genutzt werden, die expliziet für den Kunden eingerichtet wurden.
    \item \textbf{Dedicated} ist ähnlich dem privaten Ansatz mit der Einschränkung, dass sich die Server im eigenen Rechenzentrum befinden.
\end{itemize}

Bei allen drei Formen wird das Management der Server sowie der Infrastruktur vollständig von IBM übernommen. Dies gilt
sowohl für die Wartung, das Einspielen von Updates und Funktionserweiterungen als auch für den Support.

Einige Services werden in den Bluemix Varianten \textit{Private} oder \textit{Dedicated} nicht angeboten und müssen über
die Cloud bezogen werden. Dies setzt eine Verbindung ins Internet voraus.

\subsection{Katalog}
Der Bluemix Katalog umfasst alle Software-as-a-Service (SaaS) und Platform-as-a-Service (Paas) Produkte, die sich der
Benutzer selbstständig instanziieren kann. Dabei werden die Produkte in drei größere Kategorien unterteilt.

Bei \textbf{Infrastruktur} werden entweder echte oder virtualisierte Ressourcen zur Verfügung gestellt. Zum Beispiel
handelt es sich dabei um \textit{vServer} oder ein \textit{z Systems}.

Desweiteren gibt es \textbf{Apps}, welche Docker-Contrainer, Cloud Foundry Runtimes oder OpenWhisk Trigger zur Verfügung
stellen.

Die letzte große Kategorie, \textbf{Service}, beinhaltet Funktionen, welche in eine Applikation eingebunden werden können.
Über diese Funktionen werden zum Beispiel Text2Speech Funktionen für eine Anwendung bereit gestellt.

\subsection{Cloud Foundry}
Cloud Foundry ist eine Open Source Platform as a service (PaaS) Lösung. Mit Hilfe von Cloud Foundry können Entwickler
ihre Anwendungen bauen, hochladen und ausführen. Ein großer Vorteil von Cloud Foundry ist die nahezu grenzenlose
Skalierbarkeit \cite{online_grundlagen_cloudfoundry}.

Die Skalierbarkeit wird durch die Container-Architektur erreicht. Jeder Container beinhaltet eine eigene Anwendungsinstanz.
Dabei wird sowohl eine horizontale als auch eine vertikale Skalierung unterstützt.

Bei der horizontalen Skalierung werden zusätzliche Container gestartet oder gestoppt und ein Load Balancer vor die Container
geschaltet. Bei der vertikalen Skalierung werden jeder Instanz individuell z.B. mehr oder weniger Arbeitsspeicher zugeteilt.

Jede Cloud Foundry Instanz besitzt eine IPV4-Adresse wodurch es mittels A-Record (Zuordnung eines DNS-Namens zu IP-Adresse)
möglich ist, eine eigene Domain mit der Anwendung zu verknüpfen.

\section{Services}
In Bluemix gibt es mehr als 160 Funktionen. Eine großer Bereich sind die Services, welche eine
der drei Kategorien darstellt. Mit den Services können Applikationen durch vorgefertigte Funktionen erweitert werden.
Eine Bilderkennung oder Text in Sprache umwandeln sind nur wenige Beispiele.

\subsection{Secure Gateway}
Mit dem Secure Gateway können verschiedene Netzwerke miteinander verbunden werden. Dazu wird eine sichere Verbindung
zwischen Bluemix und einem anderen Netzwerk aufgebaut. Dies kann lokal im eigenen Rechenzentrum liegen oder in einer
anderen Cloud. Siehe dazu auch \cite{online_grundlagen_securegateway}.

Zur Verbindung wird ein Client im Zielnetzwerk installiert und anschließend eingerichtet.

Ein wesentlicher Vorteil des Secure Gateways ist, dass sich der installierte Client auf den Bluemix-Service verbindet und
nicht umgekehrt. Dies erleichtert die Konfiguration innerhalb des Unternehmensnetzwerkes, da z.B. keine feste IP-Adresse
vorausgesetzt wird.

\subsection{Service Broker}
Durch einen Service Broker können Cloudanwendungen geschrieben werden, welche bestehende Services erweitern oder
zusätzliche hinzufügen. Jeder Service in Bluemix besteht aus einem Service Broker.

Die Cloud Foundry Foundation hat einen offenen Standart für Schnittstellen definiert, welcher von allen größeren
Cloud-Anbietern eingehalten wird. Diesem \textit{Open Service Broker API} Projekt gehören diverse Firmen an. Mehr dazu
unter \cite{online_grundlagen_servicebrokerapi}.

Die Einrichtung eines Service Brokers in den eigenen Katalog erfolgt durch das Hinzufügen des Services in einen der zwei
zur Verfügung stehenden Scopes (Siehe dazu \cite{online_grundlagen_servicebroker}).

\subsection{Scopes}
\label{sub:scopes}
In Bluemix stehen zwei verschiedene Scopes (Bereiche) für einen Service Broker zur Verfügung.

Im \textbf{Standard Broker} kann der Service Broker für alle Endkunden oder nur für einzelne Benutzergruppen bereitgestellt
werden. In diesen Scope kann in der Regel nur der Administrator Service Broker hinzufügen.

Im \textbf{Space-Scoped Broker} steht der Service Broker nur für Mitglieder der gleichen Organisation zur Verfügung. In
diesen Scope kann jeder Benutzer einen eigenen Service Broker installieren und freigeben
(Siehe auch \cite{online_grundlagen_servicebroker_scope}).

\subsection{Toolchain}
Bei einer Toolchain handelt es sich um ein Tool zur Verwaltung von Entwicklung, Bereitstellung sowie Überwachung einer
Anwendung. Nach der Einrichtung einer Toolchain stehen verschiedene Services zur Verfügung, welche zum Beispiel eine Integration
zu einem GitHub-Projekt ermöglichen.

Mit einer eingerichteten Toolchain und einem verbundenen Git-Repository kann der Quelltext gebaut und auf einem System
installiert werden. Bei der Toolchain handelt es sich unter anderem um ein \textit{Continuous Integration}-Tool (kurz CI)
(Mehr unter \cite{online_grundlagen_toolchain}).

\section{Mainframe}
Mainframes bezeichnen leistungsfähige Großrechner, die in Rechenzentren installiert sind. Sie werden meist als
Hintergrundrechner eingesetzt, wo sie zum Beispiel für die Massendatenverarbeitung zuständig sind.

Mainframes sind immer dann sinnvoll, wenn eine Anwendung auf einem anderen Computer nicht lösbar ist oder wenn das Abarbeiten
der Anwendung auf einem herkömmlichen Computer viel zu lange dauern würde (Siehe hierzu auch \cite{book_grundlagen_mainframe}).

Die Einrichtung und Wartung eines Mainframes bedarf eines Grundverständnisses des Betriebssystems und dessen Architektur.

Neben anderen Herstellern von Mainframes gibt es IBM mit \textbf{z Systems}, welches im Folgenden näher betrachtet wird.

\subsection{z Systems}
IBM z Systems ist der Name für alle Mainframes von IBM. Der Name wurde offiziell im April 2006 veröffentlicht und von da an
ausschließlich genutzt. Vor allem durch die 64-Bit-Architektur zeichnet sich z Systems vom Vorgänger s/390 aus. Allerdings
werden ältere Programme, die noch nicht für 64-Bit angepasst wurden, ebenfalls unterstützt (Mehr in \cite{book_grundlagen_zsystems}).

\subsection{zOSMF}
IBM z/OS Management Facility (kurz zOSMF) vereinfacht die Konfiguration und unterstützt bei Managementaufgaben in einer
z/OS Umgebung. Es bietet zahlreiche optimierte Prozesse für Verwaltungsaktivitäten. Außerdem hilft es bei wiederholbaren
Aufgaben, diese zu erleichtern und zu automatisieren.

Bei zOSMF handelt es sich um eine Weboberfläche, die vom Netzwerk des Mainframes aus aufgerufen werden kann. Ein weiterer
großer Vorteil des Konfigurationsprogramms ist die REST-Schnittstelle, die es zur Verfügung stellt (Mehr über zOSMF unter
\cite{online_grundlagen_zosmf}).

Weitere Informationen und Anleitungen sind auf der zOSMF-Webseite\footnote{https://www-03.ibm.com/systems/z/os/zos/features/zosmf}
zu finden.

\subsection{CICS}
Bei Customer Information Control System (kurz CICS) handelt es sich um ein von IBM entwickeltes Programm für die
Online-Transaktionsverwaltung. Es hat sich, zusammen mit der Programmiersprache COBOL, als das weitverbreitetste Toolset
für Kundentransaktionsanwendungen im Bereich des Mainframe-Computings etabliert. Bei den meisten Legacy-Anwendungen handelt
es sich um COBOL/CICS-Programme.

Mit der API, welche CICS bereitstellt, kann ein Entwickler Anwendungen schreiben, welche mit Online-Usern kommunizieren
und Datensätze aus einer Datenbank auslesen und schreiben. Dabei werden nicht IBM-Methoden sondern CICS-Funktionen
gennutzt.

Wie andere Transaktionsmanager kann CICS sicherstellen, dass eine Transaktion korrekt abgeschlossen wurde. Falls Probleme
auftreten ist das System selbstständig in der Lage, Transaktionen rückgängig zu machen. Die Integrität von
Datensätzen bleibt dabei stets erhalten (Siehe mehr in \cite{book_grundlagen_cics} und \cite{online_grundlagen_cics}).

\section{DB2}
Bei DataBase 2 (DB2) handelt es sich um ein relationales Datenbankmanagementsystem von IBM \cite{book_grundlagen_db2}.
Seit Version 11 wird ein REST-Interface Service unterstützt, durch den eigene Schnittstellen definiert und aufgerufen
werden können.

\section{UrbanCode Deploy}
UrbanCode Deploy (kurz UCD) ist ein Tool für die Automatisierung der Anwendungsbereitstellung, das ein schnelles Feedback
und die Continuous Delivery in einer agilen Entwicklung vereinfachen und gleichzeitig die erforderlichen Versionskontrollen
und Genehmigungen für Produktionsumgebungen bereistellt.

Durch die Verwendung von UCD in Verbindung mit IBM Bluemix Private Cloud können Entwickler die Anwendungsbereitstellung
und das Management in mehreren Cloudumgebungen und Anwendungen beschleunigen.

\section{Rational Team Concert}
Bei IBM Rational Team Concert (kurz RTC) handelt es sich um eine kollaborative Lösung für das Life-Cycle-Management von
Softwareanwendungen, die auf der offenen IBM Technologieplattform Jazz basiert. Das Produkt bietet grundlegende
Projektplanung, Work Item Management, Software-Versionierung, Arbeitsbereichsverwaltung und die Unterstützung paralleler
Entwicklung.

Außerdem wurde RTC dafür ausgelegt, verteilte Entwicklungsteams zu verbinden, die Produktivität jedes Einzelnen sowie des
Teams zu steigern, die Entwicklungszyklen zu verkürzen und in kürzester Zeit qualitativ hochwertige Software bereitzustellen.

\section{WebView}
Bei WebView handelt es sich um ein Android- und iOS-Layout, das Webseiten darstellen kann. Über Schnitstellen werden
Informationen wie Webseite, URL und Größe übergeben. Das Laden, das Anzeigen und die Interaktion mit der Webseite übernimmt
das Layout selbstständig.

Das Layout steht sowohl unter Android als auch iOS in den frühesten Versionen zur Verfügung (Weitere Informationen unter
\cite{online_grundlagen_webviewer}).

\section{Serviceorientierte Architektur}
Hinter SOA, also den serviceorientierten Architekturen, steckt die Idee, die Programmlogik auf verschiedene
Services zu verteilen und sie nicht in einem einzelnen Programm zu bündeln. Prozesse werden so als eigenständige Services
definiert.

Funktionen, die durch einzelne Systeme abgedeckt werden, sind dank SOA in standardisierter Form unternehmensweit
zugänglich. Einmal erstellte Funktionen können also immer wieder verwendet werden (\cite{book_grundlagen_soa}).

Anwendungen greifen über die HTTP-Methoden wie zum Beispiel \textit{GET}, \textit{POST}, \textit{PUT}, \textit{PATCH}
und \textit{DELETE} auf die REST-Schnittstellen der Services zu und kommunizieren über \textit{JSON} oder \textit{XML}
Objekte.

\section{Model-View-Controller}
Model-View-Controller (kurz MVC) wurde um 1978 von Xerox entwickelt und es handelt sich dabei um eine Architektur von
Programmen.

In MVC wird eine Anwendungskomponente in drei Teile zerlegt. In das oberflächenunabhängige \textit{Model}, das für die
Ausgabe zuständige \textit{View} und den für die Interpretation von Eingabeereignissen zuständigen \textit{Controller}.

Der Controller ist die Steuerungseinheit, das Model der Anwendungskern und die View der Darsteller bzw. die Ansicht der
Anwendung. Ein View zusammen mit seinem Controller wird hier auch als eine Oberfläche bezeichnet.
(Mehr hierzu unter \cite{book_prototypischeanwendung_mvc})

Es gibt zahlreiche Frameworks, welche das MVC-Patern umsetzen. AngularJS ist einer der bekanntesten Vertreter.

\section{REST}
REST-API steht für Representational State Transfer - Application Programming Interface. Diese Interfaces machen den
Austausch von Informationen über unterschiedliche Systeme möglich. Im Zeitalter verteilter Systeme und Cloud-Computing
trifft man oft auf unterschiedliche Systeme, welche den Einsatz von REST-API notwendig machen.

Man spricht bei REST-API auch von der Maschine-Maschine-Kommunikation, da die verschiedenen Systeme
und Geräte zusammengebracht werden und gewissermaßen die gleiche Sprache sprechen.

Mit REST-API ist es möglich geworden, Informationen und Aufgaben auf verschiedene Systeme zu verteilen und diese mit der
Hilfe von HTTP-Requests anzufordern. Jeder HTTP-Request setzt sich aus dem Endpoint und den entsprechenden Parametern
zusammen \cite{online_grundlagen_rest}.
