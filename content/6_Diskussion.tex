\chapter{Diskussion}
\label{cha:diskussion}
In diesem Kapitel werden Themen angesprochen, die bei der Arbeit mit der Hybrid-Cloud-Architektur zu beachten sind.

Auch soll auf die Sicherheit eingegangen und Use-Cases für die Architektur vorgeschlagen werden.

\section{Rechte der Entwickler}
\label{sec:rechte_der_entwickler}
In den meisten Fällen wird dem Entwickler von einem IT-Administrator die benötigte Infrastruktur oder Ressource zur
Verfügung gestellt. Das bedeutet, dass der Entwickler diese lediglich anfordern muss, sich aber keine Gedanken darüber
macht, wo die Instanz physisch läuft. Darum kümmert sich in der Regel der IT-Administrator.

In der Hybrid-Cloud ist die Idee, dass der Entwickler sich die Ressourcen die er braucht, genau wie auf einer großen
Spielwiese, allerdings selber einrichtet.

Genau in diesem Punkt liegen auch größere Probleme. Wenn der Entwickler nun Ressourcen auf dem Mainframe oder auch in der
Cloud einrichten und seine Anwendungen in beiden Netzwerken zur Verfügung stellen kann, können unter anderem
datenschutzkritische Bestandteile in der Cloud liegen, ohne dass der Entwickler das aktiv möchte.

\section{Produktion}
Für die Entwicklung einer Anwendung ist es oftmals hilfreich einen \textit{DEV}- und ein \textit{PROD}-Stage der Anwendung
bereit zu stellen. Auch kann es von Vorteil sein, mehrere DEV-Stages zu besitzen. Darauf soll im folgenden aber nicht
näher eingegangen werden, da die Erweiterung der Stages nicht sehr schwer ist.

In der \textit{DEV}-Stage werden zum Beispiel täglich Änderungen eingebaut, die dann durch die Qualitätsabteilung
getestet und mit den benötigten Anforderungen verglichen werden können.

Wohingegen in der \textit{PROD}-Stage werden neue Updates erst gebündelt installiert, wenn sie in der DEV-Stage komplett
erfolgreich getestet und mit den Anforderungen verglichen wurden.

In der Hybrid-Cloud-Architektur gibt es mehrere Möglichkeiten, diese beiden Stages abzubilden.

In der ersten Variante wird die DEV-Stage im eigenen Rechenzentrum aufgebaut und die PROD-Stage wird in der Cloud
eingerichtet. Diese Variante macht dann Sinn, wenn die Anwendung später auch in der Cloud laufen soll.

Beim Aufbau dieser Architektur, kann die Anwendung im eigenen Unternehmen getestet werden. Die Anwendung kann ggf. mit
Daten aus der eigenen Infrastruktur gefüttert werden. Wenn die Abteilung der Qualitätssicherung eine Version dann freigibt,
kann sie in der Cloud installiert werden.

Die Verbindung zwischen der Anwendung und den Daten im eigenen Rechenzentrum wird über die Hybrid-Cloud-Architektur gelöst.

Die zweite Variante sieht die DEV-Stage in der Cloud und die PROD-Stage im eigenen Rechenzentrum vor. Dies ist immer dann
sinnvoll, wenn die Anwendung später auch im eigenen Rechenzentrum laufen soll.

Der Vorteil dieses Aufbaus liegt darin, dass die Anwendung in der DEV-Stage in der Cloud getestet werden kann. Mitarbeiter
welche diese Testen, brauchen keine VPN-Verbindung ins interne Netz um auch von zu Hause testen zu können. Nachdem eine
Version freigegeben wird, wird diese auf dem internen Server installiert und steht ab sofort zur Verfügung.

Während der Testphase kann die Anwendung mit Daten aus dem Rechenzentrum über die Hybrid-Cloud versorgt werden.

Ein weiterer Vorteil ist, dass auch externe Tester für die Anwendung herangezogen werden können. Dies ist immer dann
sinnvoll, wenn Personen, die die Anwendung gar nicht kennen, einen Blick drauf werfen sollen.

\section{Use-Cases}
Für die Hybrid-Cloud-Architektur gibt es zahlreiche Use-Cases. Im folgenden soll auf ein paar wenige eingegangen werden,
welche die bedeutsamsten Use-Cases darstellen.

\subsection{Workloads}
Manchmal ist es bei der Entwicklung einer Anwendung nicht klar, wie sie im Markt ankommen wird. Ein
Cloud orientierter Ansatz folgt dem Ansatz \glqq fail fast, fail cheaply\grqq.

Danach macht es Sinn, für die Anwendung Ressourcen in der Public Cloud bereit zu stellen, die je nach Bedarf skaliert
werden können. Somit können die enstehenden Kosten überschaubar gehalten werden.

Sollte sich über die Zeit herausstellen, dass die Anwendung von den Kunden zahlreich genutzt wird, kann sie in die
bestehende Infrastruktur integirert oder eigene Ressourcen bereitgestellt werden. Für den Übergang ist es hilfreich
eine Hybrid-Cloud aufzubauen, so können Spitzenzeiten noch abgefangen werden.

\subsection{Cloud Bursting}
Das Wort \textit{Cloud Bursting} stammt aus der Situation heraus, bei der neben der eigenen Infrastruktur eine Cloud
genutzt wurde um Kapazitätsprobleme zu umgehen. Meist handelt es sich dabei um temporäre Situationen, zum Beispiel bei
einem Online-Shop während der Weihnachtszeit.

Das \textit{Cloud Bursting} wird oft mit \glqq buy the base, rent the spike\grqq~beschrieben.

Der Hybrid-Cloud-Ansatz ermöglicht es einem Unternehmen bei auftretenden und temporären Lastspitzen eine Cloud mit zu
nutzen um diese zu bewerkstelligen. Dabei wird eine sichere und schnelle Verbindung zwischen dem eigenen und den
Cloud-Ressourcen benötigt. Dies ist mit dem Secure Gateway Service möglich.

\subsection{Hochverfügbarkeit}
Eine möglichst hohe und geografisch redundante Verfügbarkeit ist durch die eigene Infrastruktur nicht immer möglich.
Gerade kleinere und mittelständische Firmen, welche nicht in jedem Kontinent ein eigenes Rechenzentrum haben, können diese
Verfügbarkeit nicht immer gewährleisten.

Durch die Hybrid-Cloud ist es möglich, dass eigene Rechenzentrum zu nutzen und nur in Ländern in denen dieses nicht zur
Verfügung steht, die Ressourcen durch die Cloud zu erweitern.

\subsection{Katastrophenwiederherstellung}
Für Unternehmen welche eine Hybrid-Cloud-Architektur nutzen ist es zum Beispiel möglich, die eigentliche Anwendung in
der Cloud laufen zu lassen und ein Backup im eigenen Rechenzentrum. So ist es möglich, wenn die Cloud-Anwendung
nicht richtig oder gar nicht funktioniert, einfach auf die im eigenen Rechenzentrum zu springen und diese weiter zu nutzen.

Alternativ kann aus dem eigenen Rechenzentrum, in dem der Quelltext der Anwendung liegt, ein neues Deployment auf die
in der Cloud liegenden Ressourcen gestartet werden, um fehlerhafte Instanzen zu elminieren.

\subsection{Rechtliche Vorgaben}
In vielen europäischen Ländern und gerade in Deutschland ist es für viele Unternehmen wichtig, dass die gespeicherten
Daten in einem Rechenzentrum gespeichert sind, welche den hohen deutschen Datenschutzrichtlinien unterliegen. Dies bedeutet
meist, dass die Daten in Deutschland gespeichert werden müssen.

Dadurch ist es nicht möglich, die komplette Infrastruktur in die Cloud zu migrieren. Durch die Hybrid-Cloud ist es allerdings
möglich den deutschen Datenschutzrichtlinien zu entsprechen in dem die Daten im eigenen Rechenzentrum verwaltet werden
und Anwendungen, welche mit den Daten arbeiten, in eine Cloud ausgelagert werden können.

So ist es zum Beispiel möglich Anwendungen für Länder zu entwickeln und dort bereit zu stellen und trotzdem mit den Daten
zu arbeiten welche sicher in Deutschland liegen. Bei solchen Anwendungen kann es sich auch um einfache
API-Management-Anwendungen handeln, welche den Traffic an den richtigen Server schicken.