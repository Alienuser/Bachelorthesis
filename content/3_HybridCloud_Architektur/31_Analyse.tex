\section{Analyse}

\subsection{Aufgaben}
Um auf dem Mainframe verschiedene Aufgaben durchzuführen, gibt es mehrere Möglichkeiten. Eine ist das Ausführen von Kommandos
über ein Terminal. Eine andere, das Schreiben von eigenen Shell-Scripten, welche diese Aufgaben erledigen.

Alternativ können alle Aufgaben mit dem in z/OS-Version 1.13 eingeführten zOSMF durchgeführt werden. Dabei handelt es
sich um ein Web-Interface, mit dem zahlreiche Konfigurationen auf dem Mainframe vorgenommen werden können. Das Interface
besitzt darüber hinaus eine RESTFul-Schnittstelle, die es ermöglicht, eigene Programme für einzelne Aufgaben zu schreiben,
die bequem gesteuert werden können.

Ein Vorteil dabei ist, dass die geschriebenen Programme, welche die RESTFul-Schnit\-tstelle von zOSMF ansteuern, nicht auf
dem gleichen System bereit stehen müssen. Sie müssen lediglich eine Verbindung bzw. einen Kommunikationskanal zum zOSMF
haben.

\subsection{Konfigurationsinterface}
Um Ressourcen auf einem Mainframe bereit zu stellen, wird ein Interface zur Verwaltung benötigt. In dieser Arbeit stehen
zwei mögliche Interfaces zur Auswahl. Das lokale zOSMF und Bluemix, die von IBM
veröffentlichte Plattform für Cloud-Anwendungen.

Mit dem Interface soll es möglich sein, neue Ressourcen zu provisionieren, zu starten, zu stoppen und zu löschen. Außerdem
soll das Interface Informationen über die vorliegende Konfiguration und Ressourcen anzeigen können. Des Weiteren soll es
definierte Workflows in UCD starten und stoppen können.

Die Wahl fiel zugunsten von Bluemix aus, da dies öffentlich im Internet zur Verfügung steht, wohingegen zOSMF ggf. eine
VPN-Verbindung ins Netz des Mainframes voraussetzt. Außerdem enthält Bluemix, im Gegensatz zu zOSMF, die Cloud Services.

\subsection{Kommunikation}
Damit von Bluemix aus die RESTFul-Schnittstelle sowohl von zOSMF als auch von UCD angesteuert werden kann, muss eine
dauerhafte Verbindung zwischen Cloud und lokalem Netzwerk des Mainframes hergestellt werden. Dazu ist es nötig eine
VPN-"-Verbindung zwischen den beiden Netzwerken aufzubauen. Diese kann entweder manuell eingerichtet, gewartet und
gesichert werden, oder es wird der Secure Gateway Service von Bluemix genutzt.

Da sich der Secure Gateway Service selbst um die Verbindung und Authentifizierung zwischen Client und Server kümmert, wird
dieser in der Architektur genutzt. Ein weiterer Vorteil ist die leichte Konfiguration und die automatische Wiederverbindung
nach einem Neustart des Systems oder einem Verbindungsabbruch.

Außerdem können durch den Service mehrere Verbindung zwischen Cloud und Mainframe hergestellt werden, welche durch
unterschiedliche Ports realisiert werden.

\subsection{Verteilte Versionskontrolle}
Für das Speichern und Verwalten von Quelltext kommen mehrere Services in Frage. Einerseits die in Bluemix zur Verfügung
gestellten Services zur Integration von GitHub oder ein selbst gehostetes Git-Repository, alternativ auch
ein RTC, welches auf einem Mainframe installiert werden kann.

Für die Architektur ist es sinnvoll, beide Möglichkeiten zu unterstützen. So ist es möglich, den Quelltext in der Cloud
auf einem GitHub-Repository zu verwalten oder mittels RTC, lokal auf dem Mainframe.

Ein weiterer Vorteil bei der Unterstützung beider Möglichkeiten ist, dass sowohl der Quelltext, welcher in einem Cloud-Repository
verwaltet wird, durch ein Deployment auf einem lokalen Mainframe installiert, als auch der lokale Quelltext im
RTC in einer Cloud-Umgebung installiert werden kann.

\subsection{Continuous integration}
Für einen kontinuierlichen Bau eines Artefaktes aus dem Quelltext gibt es im Aufbau zwei Möglichkeiten. Entweder
den Toolchain-Service in Bluemix oder das UCD, lokal auf dem Mainframe.

Da es schon eine Verbindung zu UCD gibt und der Quelltext entweder lokal oder in der Cloud abgelegt werden kann,
werden beide Möglichkeiten umgesetzt.

So ist es möglich, mittels einer Bluemix Toolchain und verbundenem Git-Repository
das Artefakt in einer CICS-Region von der Cloud aus zu installieren oder durch das UCD, welches den Quellcode von einem RTC
lokal auf dem Mainframe abgreift.

\subsection{Qualitätssicherung}
Damit bei einem regelmäßigen Bau des Quellcodes nach einem Git-Commit die richtige Qualität gewahrt werden kann, müssen
automatisierte Tests hinterlegt werden.

Da sowohl die Bluemix Toolchain als auch UCD in der Architektur eingesetzt werden, wird in beiden Deployment-Scripten
ein Block hinzugefügt, welcher die Tests ausführt, die für die jeweilige Anwendung geschrieben wurden.