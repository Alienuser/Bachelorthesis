\section{Vorbereitung}
In diesem Abschnitt werden Vorbereitungen für die Entwicklung mit und auf Bluemix getroffen. Dabei werden unter anderem
Programme auf dem Entwicklungscomputer installiert und eingerichtet. Diese Schritte werden benötigt, da sonst nicht mit
den Systemen gearbeitet werden kann.

\subsection{Bluemix Konto}
Um mit Bluemix arbeiten zu können, wird ein kostenloses Benutzerkonto vorausgesetzt. Dieses kann über die
Registrierungs-Seite\footnote{https://console.ng.bluemix.net/registration} angelegt werden.

Ab dem Zeitpunkt der Registrierung erhält der Benutzer 30 Tage alle Services komplett kostenlos.

Nach dem erfolgreichen Anlegen des Benutzerkontos wird nach einem Namen für den ersten Bereich gefragt. Die Benamung des
Bereichs spielt dabei keine Rolle und kann frei gewählt werden. Ein Beispiel hierfür wäre \path{Hybrid Cloud}.

\subsection{Cloud Foundry CLI}
Für die weitere Konfiguration von Bluemix wird das Command Line Interface (kurz CLI) von Cloud Foundry genutzt (auch
Cloud Foundry CLI oder kurz CF genannt). Die Installation erfolgt unter Ubuntu mittels der folgenden Kommandos.

\begin{lstlisting}[language=bash, caption=Installieren der Cloud Foundry CLI, label=Installieren der Cloud Foundry CLI]
   $ wget -q -O - https://packages.cloudfoundry.org/debian/cli.cloudfoundry.org.key | sudo apt-key add -
   $ echo "deb http://packages.cloudfoundry.org/debian stable main" | sudo tee /etc/apt/sources.list.d/cloudfoundry-cli.list
   $ sudo apt-get update
   $ sudo apt-get install cf-cli
\end{lstlisting}

Ein Installer für Windows und MacOS steht im öffentlichen GitHub-Repository\footnote{https://github.com/cloudfoundry/cli}
von Cloud Foundry zur Verfügung.

Anschließend kann im Terminal, um die Installation zu überprüfen, die Version der Cloud Foundry CLI abgefragt werden. Dies
geschieht über den Parameter \path{version}.

\begin{lstlisting}[language=bash, caption=Version der Cloud Foundry CLI überprüfen, label=Version der Cloud Foundry CLI überprüfen]
   $ cf --version
\end{lstlisting}

Es sollte eine Version größer \path{6.x} angezeigt werden. Zum Beispiel \path{cf version 6.22.2+a95e24c-2016-10-27}.

Eine Übersicht über die wichtigsten Befehle der Cloud Foundry CLI ist auf der Bluemix
Hilfeseite\footnote{https://console.ng.bluemix.net/docs/cli/reference/cfcommands/index.html} zu finden.